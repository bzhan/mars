\documentclass[runningheads,a4paper]{llncs}
\usepackage{makeidx}
\usepackage{amssymb}
\usepackage{amscd, amsfonts}
\usepackage{verbatim}
\usepackage{listings}
\usepackage{commath}
\usepackage{float}
\usepackage{tikz}
\usepackage{stmaryrd}

\usepackage{lmodern}
\usepackage[T1]{fontenc}

\newcommand{\dL}{\mathsf{d}\mathcal{L}}

\lstset{
  mathescape,
  columns=fullflexible,
  basicstyle=\small\fontfamily{lmvtt}\selectfont
}

% Allow display breaks
\allowdisplaybreaks[1]

\begin{document}

\author{}
\institute{}

\title{ARCH-COMP20 Category Report: Hybrid Systems Theorem Proving}

\maketitle

\paragraph{Description of the system.}

HHL prover~\cite{WZZ15} is an interactive theorem prover for verifying
hybrid systems modelled by Hybrid CSP (HCSP)~\cite{He94,ZWR96}. HCSP is an extension of CSP by introducing differential equations for modeling continuous evolutions and interrupts for modeling interaction between continuous and discrete dynamics.  HHL prover implements the Hybrid Hoare Logic (HHL)~\cite{LLQZ10}, a Hoare style specification logic for HCSP, in the proof assistant Isabelle/HOL~\cite{isabelle}. However, as the HHL defined in~\cite{LLQZ10} is not compositional with respect to 
parallel composition, HHL prover can only handle restricted forms of parallel processes. \emph{This version of the HHL prover is used for the nonlinear models, rollercoaster, and the lunar lander control program benchmarks.}

Recently, we implemented in HHL prover a compositional trace-based specification logic for HCSP, and used it to verify some examples involving combinations of ODE, interrupt, repetition, and parallel composition. Traces for both sequential and parallel HCSP processes consist of lists of trace blocks, and describe executions of a sequential or parallel process. For sequential processes, there are three types of trace blocks: $\tau$-blocks for internal actions, input and output blocks for communication, and ODE blocks. For parallel processes, the trace blocks are $\tau$-blocks for internal action on a single process, IO blocks for communication between two processes, and wait blocks that allow waiting or ODE evolution on each process. A list of sequential traces can be combined into a parallel trace, considering the synchronization of communication events, the interleaving of discrete actions occurring at the same time, and the conjunction of ODE trajectories.

A big-step semantics is defined for sequential processes, where $(c,tr)\Rightarrow tr'$ means executing process $c$ with starting trace $tr$ may terminate with the final trace $tr'$. Hoare triples (for partial correctness) is then defined as follows:
\[\{P\}~ c ~\{Q\} \Longleftrightarrow \forall tr\ tr'.\ P(tr) \longrightarrow (c,tr)\Rightarrow tr' \longrightarrow Q(tr') \]

We defined inference rules for Hoare triples based on traces, which enables reasoning about HCSP processes in a compositional way. \emph{The new system is tested on a benchmark consisting of small HCSP programs involving combinations of ODE, interrupt, repetition, and parallel composition.}
 


\paragraph{Example: design shapes and nonlinear continuous models.}

We translated some of the examples in the first two parts of the
benchmarks to HCSP. There are some essential differences between the
semantics of $\dL$ and HCSP. In particular, in HCSP, evolution by a
differential equation cannot stop before reaching the boundary (or
interrupted by a communication). Also, there is no assignment to an
arbitrary value ($x := *$) in HCSP. As a result, some of the examples
cannot be translated naturally. For those examples that can be
translated naturally, we proved some of them in Isabelle, with the
help of invariant checking using Redlog.

\paragraph{Example: Rollercoaster.}

We converted the rollercoaster example \cite{coasterx} to HCSP. The
conversion is natural, as the differences between $\dL$ and HCSP do
not produce any problems. The proof makes use of invariant checking
using Redlog, as well as the newly added differential ghost rule. The
entire Isabelle theory (including the model, specification, and proof
for all ten parts of the example) is 1141 lines long.

\paragraph{Example: lunar lander control program.}

The lunar lander control program is a closed loop system, which is
composed of the lander's dynamics and the guidance program for the
slow descent phase. The guidance program is executed periodically with
a fixed sampling period. At each sampling point, the current state of
the lander is measured by inertial measurement unit or various
sensors. Processed measurements are then input into the guidance
program, which outputs control commands, e.g. the magnitude and
direction of thrust, to be imposed on the lander's dynamics in the
following sampling cycle.

The mathematical description of the lander's dynamics as well as the
guidance program of the slow descent phase can be found
in~\cite{ZYZG14,ZhanWZ16}. The entire Isabelle theory including the
model, specification, and proof for the entire example is 327 lines
long. By applying HHL prover, the unproven subgoals related to
differential invariants are transformed to a set of SOS constraints
with respect to the user-defined invariant template, and then the
SOS-based invariant generator is invoked on these constraints to
synthesize a satisfying invariant.

\paragraph{Example: benchmarks for trace-based logic}

We choose some examples of sequential and parallel processes involving combinations of ODE, interrupt, repetition, and parallel composition, and verified their properties using the trace-based logic for HCSP in Isabelle. In case of processes containing repetition, the invariant is an inductively defined assertion on traces, describing the set of traces that can occur after any finite number of repetitions. Defining and reasoning about such inductively defined assertions is one of the main challenges of the verification. We also build upon the existing analysis and ODE library in Isabelle to provide foundational proofs about properties of ODE evolution.

\bibliographystyle{splncs04}
\bibliography{main}

\end{document}
