\documentclass{llncs}
\pagestyle{plain}

\usepackage{amsmath,amsfonts,amssymb}
\usepackage{stmaryrd}
\usepackage{graphpap}
\usepackage{color}
\usepackage{listings}
\newcommand{\mycomment}[1]{{\color{red}[#1]} }


\lstdefinelanguage{isa} {alsoletter={\#, \&, $},
  mathescape=true,
  basicstyle=\small,
  escapechar=\@,
  boxpos=c,
  morekeywords= {and, axioms, axiomatization, Skip, Stop, type_synonym, theorem, lemma, apply, by, constdefs, definition, where,
    infixl, types, consts, primrec, have, from, show, let,  proof, qed, is,
    sorry, with, assume, fix, thus, hence, datatype, if, then, else, in, case,
    of
  },
  emph={[2] real, int, char, bool, string, self, boolean, Ass, Send, Rec, Seq, Cond, Pref,
                Join, Meet, Par, Rep, Cont, TOut, Inrp},
  emphstyle={[2]\it},
literate=
  {->}{{$\rightarrow$}}2
  {=>}{{$\Rightarrow$}}2
  {-->}{{$\Rightarrow$}}2
  {\\forall}{{$\forall$}}2
  {\\exist}{{$\exists$}}2
  {\\exists}{{$\exists$}}2
  {AND}{{$\& \&$}}3
  {ALL}{{$\forall$}}2
  {EX}{{$\exists$}}2
  {\%}{{$\lambda$}}1
  {\\\/}{{$\sqcap$}}2
  {|-}{{$\vdash$}}2
  {==}{{$\equiv$}}2
  {==>}{{$\Rightarrow$}}2
  {~}{{$\neg$}}1
  {~=}{{$\neq$}}2
  {:=}{{\tt :=}}2
  {;}{{\tt ;}}1
  {$}{{\do}}1
}

\lstnewenvironment{isaenv} {
  \lstset{extendedchars=true,columns=flexible}
  \lstset{language=isa,deletestring=[b]', basicstyle=\footnotesize}%
} {}
\newcommand{\isa}[1]{{\lstinline[language=isa, basicstyle={\footnotesize \tt} ]+#1+}}

\newcommand{\Var}{\mathcal{V}}
\newcommand{\Chan}{\mathcal{C}}


% Constructors of HCSP
\newcommand{\pskip}{\textmd{skip}}
\newcommand{\pstop}{\varepsilon}
\newcommand{\evofun}[2]{\mathcal{#1}(\dot{#2},#2)=0}
\newcommand{\evolution}[3]{\langle \mathcal{#1}(\dot{#2},#2)=0 \& #3\rangle}
\newcommand{\evo}[3]{\langle \dot{#1}=#2\& #3\rangle}
\newcommand{\evon}[2]{\langle \dot{#1}=#2\rangle}
\newcommand{\imitate}[4]{\langle \mathcal{#1}(\dot{#2},#2)=0; #3 \& #4\rangle}
\newcommand{\pwait}{\textrm{wait}}
\newcommand{\exempt}[4]{#1 \unrhd \talloblong_{#2} (#3 \rightarrow #4)}
\newcommand{\exemptS}[3]{#1 \unrhd (#2 \rightarrow #3)}
\newcommand{\external}[3]{\talloblong_{#1} (#2 \rightarrow #3)}
\newcommand{\internal}[3]{\sqcup_{#1} (#2 \rightarrow #3)}
\newcommand{\lr}{\talloblong}
\newcommand{\fracN}[2]{\frac{\textstyle #1}{\textstyle #2}}
\newcommand{\almost}[1]{\lceil #1 \rceil}
\newcommand{\specF}[4]{\{#2\} #1 \{#3; \; #4\}}
\newcommand{\chop}{\smallfrown}
\newcommand{\close}{\textit{cl}}
\newcommand{\chanset}{\textit{chans}}

 

% Liang's notation
\newcommand{\pdef}{\ \widehat{=}\ }
\newcommand{\st}{\, . \, }
\newcommand{\angles}[1]{\langle #1 \rangle}
\newcommand{\prepeat}[1]{#1^*}
\newcommand{\pindent}{\qquad}
\newcommand{\trainefun}[1]{\ensuremath{\angles{\dot{s}=v,\dot{v}=a,\dot{t}=1 \&\ #1}}}
% Jargon
\newcommand{\djzhd}{\zl{dengji zhuanhuan dian}\ }
% Zou Liang's emphasis
\newcommand{\zl}[1]{\textcolor{blue}{#1}}

\newcommand{\oomit}[1]{}
\newcommand{\Int}{\mbox{$\int$}}
\def \dur {\mbox{$\int\!$}}

\newcommand{\chopp}{{}^\chop}

\newcommand{\leadm}[1]{\xrightarrow{#1}}
\newcommand{\seman}[1]{[\![#1 ]\!]}
\newcommand{\updateh}[4]{#1 \langle #2, #3, #4 \rangle}
\newcommand{\ommit}[1]{}

\begin{document}

\mainmatter

\title{A Trace Based Logic for Hybrid CSP in Isabelle}
\author{ }
\institute{ }


\maketitle

 \section{The Syntax of HCSP}
 
 The syntax of a subset of HCSP is given as follows:
\[
\begin{array}{lll}
 P  & ::= & \pskip \mid x :=e  \mid ch?x \mid ch!e  \mid P;Q  \mid B \rightarrow P\mid  P \sqcup Q \mid P^*\\
      && \mid \evo{s}{e}{B} \mid \exempt{\evon{s}{e}}{i\in I}{io_i}{Q_i}\\
 S &::=& P \mid S\|S
  \end{array}
 \]
where $P, Q, Q_i, S$ are HCSP processes, $x$ and $s$ stand for process variables,
 $ch$ for channel name, $io_i$ for a communication event (either input $ch?x$ or output  $ch!e$),
$B$ and $e$ for Boolean and arithmetic expressions,
 and $I$ for a non-empty set of indices of communications, respectively.
 A whole HCSP model $S$ is defined
 as a sequential process or a parallel composition of several sequential processes at the top level.

The intuitive meaning of the individual constructs is explained as follows:
\begin{itemize}
\item  $\pskip$, $x: = e$ are defined as usual.
%terminates immediately having no effect on variables.
%\item  $x: = e$ assigns the value of expression $e$ to $x$ and then terminates.
\item The input $ch?x$ receives a value along channel $ch$ and assigns it to $x$, and output
 $ch!e$ sends the value of $e$ along  $ch$.
A communication takes place as soon as both the sending party (i.e. $ch!$) and the receiving party (i.e. $ch?$) are ready, and may cause one side to wait.
\item The sequential composition $P; Q$ behaves as $P$ first, and if it terminates, as $Q$ afterwards.
\item The conditional $B \rightarrow P$ behaves as $P$ if $B$ is true, otherwise it terminates immediately.
\item The internal choice $P \sqcup Q$ behaves as either $P$ or $Q$, and the non-deterministic choice is made by the system itself.
\item The repetition $P^*$ executes $P$ for some finite number of times.
\item $\evo{s}{e}{B}$ is the continuous
 evolution statement, where $s$ represents a vector of real variables and $\dot{s}$ the first-order derivative of $s$.
 It forces  $s$  to evolve continuously according to  the differential equation  $\dot{s}=e$ as long
 as $B$, which defines the
{\em domain of $s$}, holds, and terminates when $B$ turns false. $\evo{s}{e}{B}$ is a boundary interruption.
\item The communication interruption $\exempt{\evo{s}{e}{B}}{i\in I}{io_i}{Q_i}$ behaves like the continuous $\evo{s}{e}{B}$, except that it is preempted as soon as one of the communications $io_i$ takes place,
and then is followed by the respective $Q_i$.
% Notice that, if the continuous terminates before a communication from among $\{io_i\}_{i\in I}$  occurs, then the process terminates immediately without waiting for communication.
\item $S_1\|S_2$ behaves as if $S_1$ and $S_2$ run independently except that all communications along the common channels connecting $S_1$ and $S_2$ are to be synchronized. $S_1$ and $S_2$ in parallel can neither share variables, nor input or output channels.
\end{itemize}
 
 \section{The Trace-based Semantics of HCSP}
 
 \subsection{Trace Blocks}
 
 To present the trace-based semantics of HCSP, we introduce some notations below. A state $s$ maps variables to values, and it keeps changing according to the execution of processes.
 We define \isa{rdy} as  a pair of input and output channel names $(InC, OutC)$, meaning that the input events along channels in $InC$ and the output events along channels in $OutC$ become ready, but need to wait for their respective partner event for a synchronized communication.   Two ready sets $(In1, Out1)$ and $(In2, Out2)$ are compatible, iff
 $In1 \cap Out2 = Out1 \cap In2 = \emptyset$, i.e. there will never be the case when an input and an output along a same channel both wait for a positive time interval. 
 
 
 
 A trace block \isa{trace_block} defines the atomic execution behavior for different HCSP processes. It 
 includes the following types: 
 \begin{itemize}
   \item \isa{InBlock d ch x v rdy} represents that the input event along channel $ch$ waits $d$ time units till the communication can occur, and at time $d$, the value $v$ sent from the partner is assigned to receiving variable $x$;  during the waiting time, the  channels in $rdy$ keep waiting. 
  \item The output block \isa{OutBlock d ch v rdy} is similar as input, except that no receiving variable is recorded.
  \item  \isa{TauBlock s} represents that an internal discrete action occurs, changing the state to $s$.
  \item \isa{WaitBlock d} represents that a time progress of length $d$ occurs, with no variable being changed.
  \item \isa{ODEBlock d h} represents a continuous time evolution for $d$ time units, resulting in an execution history $h$, which
 maps each time in the interval $[0,d]$ to a state.
 \item \isa{ODEInBlock d h ch x v rdy} represents an input communication interrupt  to an ODE, where $d$ is the evolution time and $h$ the corresponding execution history for ODE; $ch$ is the input channel, $x$ receiving variable, $v$ sent from outside, and $rdy$ the ready channels. 
\item \isa{ODEOutBlock d h ch v rdy} represents an output communication interrupt to an ODE, similar as above case. 
 
 \end{itemize}
 
 With the definition of trace blocks, a trace is defined as a combination of an initial state and a sequence of trace blocks (implemented as a list in Isabelle). We use functions \isa{start_of_trace} and \isa{blocks_of_trace} to return them respectively. The behavior of each sequential process can be defined as a trace.
 For manipulating trace blocks, we define several functions. Given a state \isa{s} and a trace block list \isa{blks}, we define function
 \isa{end_of_blocks s blks} to represent the terminated state after the execution of \isa{blks} from state \isa{s}. This function can be defined recursively, for which the terminated state after executing each atomic trace block is defined clearly in their definitions.  Given a trace block list \isa{blks} and time $t$, function \isa{rdy_of_block blks t} returns the set of ready channels at time $t$. 
 
  Above we define trace blocks for recording the trace behavior of sequential processes, next we introduce \isa{par_trace} to record the trace for parallel processes. Currently, we only consider closed parallel processes, for which no external communication is allowed. In general, we assume there are \isa{n} processes in parallel, denoted by \isa{1, 2, ..., n} respectively.
 A parallel trace block \isa{par_block} includes three types: 
 \begin{itemize}
   \item \isa{InBlock i j ch x v} represents that process \isa{i} and \isa{j} perform a synchronized communication along channel \isa{ch} together, by assigning value \isa{v} to variable \isa{x}.
       
       \item \isa{ParTauBlock i s} represents that process \isa{i} preforms an internal discrete action, resulting its state in \isa{s}.
           
           \item \isa{ParWaitBlock d hh} represents that all the processes in parallel perform a continuous time evolution for \isa{d} time units and produce a history \isa{hh}. \isa{hh} maps each time in \isa{[0, d]} to a state list corresponding to the processes. 
               
  \end{itemize}
 With the definition of parallel trace blocks, a parallel trace is defined as a combination of a sequence of initial states and a sequence of parallel trace blocks (both implemented as lists in Isabelle). The state list records the initial states for all the sub-processes in parallel.
 
\oomit{ We define \isa{end_of_par_blocks sts pblks} to return the terminated state list after the execution of \isa{pblks} from state list  \isa{sts}.} 

\paragraph{\textbf{\emph{Parallel Composition of Trace Blocks}}}
For a sequential process, we can use an initial state \isa{s} and a trace block list \isa{blks} to represent its trace. 
Now we define a function \isa{combine_blocks} for the parallel composition of traces, which takes a state list (i.e. the initial states for the sequential processes) and a list of trace block lists (i.e. the trace block list for the sequential processes) as input, and returns a parallel trace block list as output. At each time, the trace blocks for different processes are combined depending on their types. There are three cases:
\begin{itemize}
  \item For current time, when there is an available \isa{TauBlock} for some process \isa{i}, then a \isa{ParTauBlock} on \isa{i} executes. Then, the top internal block is removed from the trace block list of process \isa{i}, and the combination of the residual trace blocks is considered recursively.
  \item When there are two  processes \isa{i} and \isa{j}, one with an input event being ready, and the other with an output event along the same channel being ready, then a communication block occurs on \isa{i} and \isa{j}. Then the top input and output blocks are removed from the trace block lists of \isa{i} and \isa{j}, and  the combination of the residual trace blocks is considered recursively.
  \item When for all trace block lists, the top trace blocks have delay time greater than \isa{t}, then a \isa{ParWaitBlock} occurs on all the processes. In consequence, a wait block of length \isa{t} is removed from the trace block list of each process, then the combination of the residual trace blocks is considered recursively.
\end{itemize}
 
 
 
 \subsection{Semantics}
 
 For a sequential process $P$, its semantics is defined as a mapping from traces to traces. We pick some rules to illustrate the semantics. 
 
 For output event $ch!e$, its execution produces an outblock and it is attached to the tail of trace $tr$ as follows: 
 \[(ch!e, tr)  \rightarrow tr \dagger \isa{OutBlock d ch v (\{ch\}, \{\})}\]
 where $d \geq 0$ is an arbitrary delay time, $v$ is the value of $e$ under initial trace $tr$. During the delay time, 
 only the input channel $ch$ is waiting in the ready set. 
 
 
 For assignment $x:=e$, it produces an internal block:
 
 \[(x:=e, tr)  \rightarrow tr \dagger \isa{TauBlock }\ s[x \mapsto v]\]
 where $s$ is the state corresponding to $tr$ and $v$ is the value of $e$ at $s$. 
 

 
 
  \section{Specification Logic and Inference Rules} 

\end{document}
