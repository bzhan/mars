\documentclass[runningheads,a4paper]{llncs}
\usepackage{makeidx}
\usepackage{amssymb}
\usepackage{amscd, amsfonts}
\usepackage{verbatim}
\usepackage{listings}
\usepackage{commath}
\usepackage{float}
\usepackage{tikz}
\usepackage{stmaryrd}

\usepackage{lmodern}
\usepackage[T1]{fontenc}

\newcommand{\dL}{\mathsf{d}\mathcal{L}}

\lstset{
  mathescape,
  columns=fullflexible,
  basicstyle=\small\fontfamily{lmvtt}\selectfont
}

% Allow display breaks
\allowdisplaybreaks[1]

\begin{document}

\author{}
\institute{}

\title{ARCH-COMP20 Category Report: Hybrid Systems Theorem Proving}

\maketitle

\paragraph{Description of the system.}

HHL prover~\cite{WZZ15} is an interactive theorem prover for verifying
hybrid systems modelled by Hybrid CSP (HCSP)~\cite{He94,ZWR96}. HCSP is an extension of CSP by introducing differential equations for modeling continuous evolutions and interrupts for modeling interaction between continuous and discrete dynamics.  HHL prover
implements the Hybrid Hoare Logic (HHL)~\cite{LLQZ10}, a Hoare style specification
logic for HCSP, in the proof assistant
Isabelle/HOL~\cite{isabelle}. However,  as the HHL defined in~\cite{LLQZ10} is not compositional with respect to 
parallel composition, HHL prover can only handle restricted forms of parallel processes. 

Recently, HHL prover implements a compositional trace-based specification logic for HCSP, and has
been used for verifying some examples involved with communications, parallel composition, repetition, and so on. 
In the trace logic, the behavior of a HCSP process $P$ is specified by predicates on traces,
\[Valid\  \{P\}~ c ~\{Q\} \Leftrightarrow (\forall tr, tr'. P\ tr \rightarrow Bigstep \ c \ tr\ tr' \rightarrow Q\ tr') \]
saying that, if precondition $P$ holds for initial trace $tr$, and if the execution of $c$ terminates and  changes $tr$ to $tr'$, then the postcondition $Q$ holds for $tr'$. The trace of both sequential and parallel HCSP processes here is a list of atomic trace blocks. For sequential processes, there are three different types of trace blocks: the Tau block specifies discrete internal actions such as skip, assignment, that are executed in zero time; the input and output blocks specify the readiness and the value transmitted of corresponding communication events; the ODE blocks specifies the time duration and the trajectory for the ODE continuous evolution. Especially, the communication interrupt to continuous evolution can be specified by combining the last two types of trace blocks.   
 For parallel processes of HCSP, the trace can be defined as the combination  of the traces of their sub-processes. The combined trace considers the time and value synchronization of communication events, the interleaving of discrete actions occurring at the same time, and the conjunction of ODE trajectories throughout the common time intervals. Based on the trace notations, HHL defines the inference rules for all HCSP processes in a compositional way.
 
 






 
 
 

\paragraph{Example: design shapes and nonlinear continuous models.}

We translated some of the examples in the first two parts of the
benchmarks to HCSP. There are some essential differences between the
semantics of $\dL$ and HCSP. In particular, in HCSP, evolution by a
differential equation cannot stop before reaching the boundary (or
interrupted by a communication). Also, there is no assignment to an
arbitrary value ($x := *$) in HCSP. As a result, some of the examples
cannot be translated naturally. For those examples that can be
translated naturally, we proved some of them in Isabelle, with the
help of invariant checking using Redlog (details to be completed by
Mar 24).

\paragraph{Example: Roller coaster.}

We converted the roller coaster example \cite{coasterx} to HCSP. The
conversion is natural, as the differences between $\dL$ and HCSP does
not produce any problems. The proof makes use of invariant checking
using Redlog, as well as the newly added differential ghost rule. The
entire Isabelle theory (including the model, specification, and proof
for all ten parts of the example) is 1141 lines long.

\paragraph{Example: lunar lander control program.}

The lunar lander control program is a closed loop system, which is
composed of the lander's dynamics and the guidance program for the
slow descent phase. The guidance program is executed periodically with
a fixed sampling period. At each sampling point, the current state of
the lander is measured by inertial measurement unit or various
sensors. Processed measurements are then input into the guidance
program, which outputs control commands, e.g. the magnitude and
direction of thrust, to be imposed on the lander's dynamics in the
following sampling cycle.

The mathematical description of the lander's dynamics as well as the
guidance program of the slow descent phase can be found
in~\cite{ZYZG14,ZhanWZ16}. The entire Isabelle theory including the
model, specification, and proof for the entire example is 327 lines
long. By applying HHL prover, the unproven subgoals related to
differential invariants are transformed to a set of SOS constraints
with respect to the user-defined invariant template, and then the
SOS-based invariant generator is invoked on these constraints to
synthesize a satisfying invariant.


\bibliographystyle{splncs04}
\bibliography{main}

\end{document}
