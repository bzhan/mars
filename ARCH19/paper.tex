\documentclass[runningheads,a4paper]{llncs}
\usepackage{makeidx}
\usepackage{amssymb}
\usepackage{amscd, amsfonts}
\usepackage{verbatim}
\usepackage{listings}
\usepackage{commath}
\usepackage{float}
\usepackage{tikz}
\usepackage{stmaryrd}
\usepackage{isabelle}
\usepackage{isabellesym}

\usepackage{lmodern}
\usepackage[T1]{fontenc}

\isabellestyle{tt}

\newcommand{\isato}{\isasymrightarrow}
\newcommand{\isaTo}{\isasymRightarrow}
\newcommand{\isain}{\isasymin}
\newcommand{\isaiff}{\isasymlongleftrightarrow}
\newcommand{\isaand}{\isasymand}
\newcommand{\tr}{\operatorname{tr}}

\lstset{
  mathescape,
  columns=fullflexible,
  basicstyle=\small\fontfamily{lmvtt}\selectfont
}

% Allow display breaks
\allowdisplaybreaks[1]

\begin{document}

\author{}
\institute{}

\title{ARCH-COMP19 Category Report: Hybrid Systems Theorem Proving}

\maketitle

\begin{abstract}
  Report for HHL Prover
\end{abstract}

HHL prover~\cite{WZZ15} is an interactive theorem prover for verifying hybrid systems modelled by Hybrid CSP (HCSP)~\cite{He94,ZWR96}.
It implements the Hybrid Hoare Logic (HHL)~\cite{LLQZ10}, that is a Hoare style logic for reasoning about HCSP,  in proof assistant Isabelle/HOL~\cite{isabelle}.

HCSP is an extension of CSP, by  introducing continuous variables, differential equations, and interruptions by boundary, timeout and communication.
Given a HCSP process $P$,  a HHL specification takes the form $\{Pre\}~P~\{Post; HF\}$, where
$Pre$ and $Post$ in first-order logic are pre-/post-conditions, and $HF$ in duration calculus~\cite{ZH04} is history formula
specifying the  properties held throughout the whole execution interval. HHL defines a set of axioms and inference rules for deducing HCSP specifications. HHL prover formalizes HCSP and HHL for proving the partial correctness of hybrid systems.

For proving differential equations, 



\begin{itemize}
\item Introduction to HHL Prover (shorter than last year).



\item Ghost rule and cut rule (need to account for the differences
  between dL and HCSP).
\item Verification of differential invariants using quantifier
  elimination in redlog (as an oracle in Isabelle).
\item Invariant generation for the lunar lander example (by Mar 24,
  removed dependency on Mathematica).
\item Examples:
  \begin{itemize}
  \item Design shapes and nonlinear continuous models (translated some
    examples to HCSP, finish by Mar 24).
  \item Lunar lander example (case study benchmarks, using invariant
    generation).
  \item Rollercoaster example (case study benchmarks, invariant
    checking).
  \end{itemize}
\end{itemize}

\bibliographystyle{splncs04}
\bibliography{main}

\end{document}
